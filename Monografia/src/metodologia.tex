\section{Metodologia}

A metodologia utilizada neste trabalho é uma proposta de metodologia de projeto
para aplicações utilizando Bluetooth Low Energy. 

Para exemplificar a utilização do método, o projeto feito é implementado numa
plataforma de prototipagem para a validação da metodologia.

\subsection{Projeto de aplicação com Bluetooth Low Energy}

A primeira etapa do projeto do dispositivo BLE consiste na definição de todas
tarefas que deverão ser realizadas pelo dispositivo. Definidas as tarefas, é
escolhida uma técnica para a troca de informações com o dispositivo, isso
implica em definir quais tarefas o dispositivo irá realizar como broadcaster,
como slave, como scanner, como central e quais tarefas são independentes do
bluetooth, como por exemplo acesso a sensores ou atuadores.

O projeto das tarefas relacionadas ao bluetooth pode ser feito sem especificar
os componentes de hardware necessário, porém é interessante selecionar os
componentes nesta etapa do projeto. Desta forma é possível saber de antemão
quais são os recursos e informações que podem ser disponibilizados.

Ao trabalhar com projetos que envolvem baixo consumo de energia ou projetos com
energia limitada, o consumo energético dos componentes se torna uma
característica importante, necessitando a análise dos datasheets de cada
componente buscando sempre um consumo energético aceitável.

Em relação ao bluetooth deve-se projetar cada forma de operação do dispositivo e
os parâmetros comuns de operação.

No projeto da operação como broadcaster deve-se definir como serão codificadas
as informações atribuídas ao broadcaster nos pacotes de advertising. Além
disso, a especificação suplementar especifica uma série de informações
pré-definidas que podem ser transmitidas no mesmo pacote de advertising emitido
pelo broadcaster\cite{ble4sup}. Pode ser necessária também a especificação do
pacote de scan response, que deve seguir as mesmas especificações do
advertising. Ao término desta definição, é possível prever quantos tipos
diferentes de pacotes o dispositivo deverá transmitir.

No projeto da operação como slave, deve-se definir todos os serviços bluetooth
que o dispositivo deve oferecer, bem como as suas respectivas características. 
No serviço, define-se o UUID único do serviço. Já para as características, é
necessário definir o UUID, os tamanhos mínimos e máximos da característica em
bytes, as permissões de leitura e escrita, o envio de notificações, o envio de
alarmes e os descritores das características.

No projeto da operação como observer, deve-se definir a janela de
amostragem, o intervalo entre amostragens e quais são os pacotes que devem ser
processados pelo dispositivo.

No projeto da operação como central, deve-se definir quais são os dispositivos
alvo da conexão, quais são os serviços e caracteríticas que o dispositivo deve
acessar, ler e escrever, bem como definir quais são os dados escritos. É
importante notar que para operar como central o dispositivo também deve operar
como observer com a finalidade de identificar os dispositivos alvo. É
necessário também saber quais são os serviços disponíveis nos dispositivos alvo.

Pode ser necessária uma ou mais iterações entre as etapas de projeto com a
finalidade compatibilizar todos os elementos do projeto.

\subsection{Implementação e verificação do projeto}

A implementação do dispositivo se deu através da construção do hardware
necessário para a integração dos componentes e da elaboração do software
embarcado responsável pelo gerenciamento dos sensores e de todas as operações
realizadas pelo bluetooth, e pela verificação das funcionalidades do
dispositivo.

\subsubsection{Montagem do Hardware}
\subparagraph{Material Utilizado}

\begin{itemize}[noitemsep]
  \item Kit de desenvolvimento BLE400 para nRF51822
  \item Kit de desenvolvimento ESP32-DevKitC para ESP32
  \item Breakout Board do sensor BMP180
  \item Breakout Board do sensor HTU21D
  \item Breakout Board do sensor TSL2561
  \item Gravador OB JLink V8
  \item Placa padrão
  \item Estanho para solda liga 60\%Sn/40\%Pb
\end{itemize}

A construção do hardware ocorreu nas seguintes etapas: compra dos
componentes, definição dos pinos da comunicação entre os componentes e soldagem
dos componentes em placa padrão com conectores para efetuar as ligações.


\subsection{Elaboração do Software Embarcado}

\subparagraph{Material Utilizado}
\begin{itemize}[noitemsep]
  \item Software Eclipse IDE versão Oxygen
  \item NRF5 Software Development Kit versão 12.3
  \item Espressif IoT Development Framework versão 3.1
  \item Serviço em núvem Amazon AWS IoT Core
  \item Make
  \item GNU Arm Embedded Toolchain versão 6.3.1
  \item Xtensa GCC Toolchain versão 5.2.0
  \item Ferramenta NRFJPROG versão 9.7.2
  \item Ferramenta ESPTOOL versão 2.2.1
  \item SEGGER JLink Software versão 6.12f
\end{itemize}

A elaboração do software embarcado se deu nas seguintes etapas: teste do
ambiente de desenvolvimento com os exemplos do fabricante da plataforma,
definição dos componentes de software e arquitetura, elaboração e testes das
bibliotecas de acesso aos sensores em linguagem C com base nos datasheets dos
componentes, elaboração dos demais componentes de software em linguagem C e
integração de todos os programas desenvolvidos.

\subsection{Verificação das Funcionalidades}

\subparagraph{Material Utilizado}
\begin{itemize}[noitemsep]
  \item Aplicativo para smartphone Android nRF Connect versão 4.19.1
  \item Software para terminal serial Minicom versão 2.7 para sistema
  operacional Linux
  \item Navegador de internet Mozilla Firefox versão 58.0.2
\end{itemize}

Através do aplicativo nRF Connect foi possível testar as funcionalidades
relacionadas ao bluetooth broadcaster, observando e analisando os pacotes de
advertisement enviados, e ao bluetooth slave, estabelecendo a conexão,
realizando a leitura dos sensores, habilitando o streaming de dados e testando
novas configurações no dispositivo.

As funcionalidades do bluetooth observer foram testadas através do terminal
serial, enviando os pacotes recebidos através da porta UART para um computador
que mostra as mensagens recebidas.

O envio de mensagens para a internet foi testado através do dashboard online
oferecido pela Amazon AWS que permite o monitoramento do projeto através do
navegador de internet.

Também foram realizados testes de performance energética no dispositivo através
de um medidor de corrente do tipo shunt alimentando a placa, com a leitura do
medidor ligada a um osciloscópio digital.


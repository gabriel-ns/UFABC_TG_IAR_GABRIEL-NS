\section{Conclusões e Perspectivas}

% a metodologia se mostrou eficaz
Utilizar a metodologia proposta se mostrou eficaz na organização da
implementação do projeto por já ter pré definida a estrutura da aplicação
bluetooth, servindo como referência para o desenvolvimento do projeto para os
dois dispositivos. 

% ambientes de desenvolvimento
Os ambientes de desenvolvimento montados e utilizados ao longo do projeto
cumpriram suas funções, sendo possível desenvolver e compilar os programas
livremente. As ferramentas de depuração para o microcontrolador nRF51822 foram
úteis para agilizar o desenvolvimento, sendo usada para a realização dos
testes dos drivers dos sensores e detectar rapidamente em qual operação ocorriam
erros. Já para o ESP32 o ambiente não era capaz de oferecer a depuração do
programa, o que dificultaria o processo de desenvolvimento de softwares mais
complexos e com mais componentes. Durante o projeto, não se descobriu uma
alternativa para a depuração do programa para o ESP32, e como solução utiliza-se
o log de mensagens para se verificar as sequências de operações e os valores de
variáveis. 

% estação de medidas foi completamente funcional
A estação de medidas desenvolvida se mostrou funcional para o propósito. Numa
próxima etapa é possível trabalhar as questões relacionadas a eficiência
energética do dispositivo, estimativas de duração de bateria, e até mesmo
desenvolver um hardware próprio independente de kits de desenvolvimento.

Durante os testes foi possível notar também que os dados transmitidos pela
estação de medidas podem ser perdidos pelos dispositivos que os recebem,
especialmente nos smartphones em que não há acesso ao controle da janela de
amostragem. Este tipo de perde evidencia que o sistema atende os requisitos
apenas para monitoramento, porém é necessário elevar a confiabilidade da entrega
das medidas para utilizar esta técnica em sistemas de controle.

Com este trabalho espera-se oferecer a comunidade acadêmica um primeiro passo no
entendimento e desenvolvimento de aplicações utilizando o Bluetooth Low Energy,
oferecendo a explicação, um exemplo de utilização e um projeto de referência
detalhado para microcontroladores com processadores ARM Cortex.

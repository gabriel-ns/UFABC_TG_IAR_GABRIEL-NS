\section{Conclusões e Perspectivas}

% a metodologia se mostrou eficaz


% estação de medidas foi completamente funcional
A estação de medidas desenvolvida se mostrou funcional para o propósito. Numa
próxima etapa é possível trabalhar as questões relacionadas a eficiência
energética do dispositivo, estimativas de duração de bateria, e até mesmo
desenvolver um hardware próprio independente de kits de desenvolvimento.

Durante os testes foi possível notar também que os dados transmitidos pela
estação de medidas podem ser perdidos pelos dispositivos que os recebem,
especialmente nos smartphones em que não há acesso ao controle da janela de
amostragem. Este tipo de perde evidencia que o sistema atende os requisitos
apenas para monitoramento, porém é necessário elevar a confiabilidade da entrega
das medidas para utilizar esta técnica em sistemas de controle.


% dados chegaram na internet
	% há a possibilidade de perda de pacotes
	% custo da infraestrutura online

%PERSPECTIVAS

% que o trabalho sirva de referência
% que o trabalho contribua com o passo inicial para o desenvolvimento de mais
% aplicações com BLE
% contribuir com os primeros passos na programação de microcontroladores ARM de
% 32 bits


\section{Desenvolvimento}

\subsection{Projeto}

O projeto desenvolvido tem como funcionalidade principal a coleta de leituras
periódicas de sensores com a transmissão sem fio das variáveis medidas
utilizando o bluetooth low energy. Estes dados, quando recebidos, são enviados
para a internet.

Para tal, dividiu-se as tarefas em três sistemas diferentes: a estação de
medidas, responsável pela coleta e transmissão das medididas através do
bluetooth, o gateway BLE-Internet, responsável por captar os dados no ar e
enviá-los para a internet, e a infraestrutura online, que recebe a informação e
a armazena on-line.

\subsubsection{Estação de Medidas Bluetooth Low Energy}

A estação de medidas foi projetada para atender os requisitos propostos,
visando o mínimo consumo energético para possibilitar a alimentação através de
baterias sem a necessidade de manutenções frequentes.
 
Como diferencial a estação permite a configuração de sua operação através de
serviços GATT do BLE, sendo possível alterar tanto as configurações de rádio
(potêcia e intervalo de transmissão) quanto os parâmetros de configuração dos
sensores (range, escala, intervalo entre medidas).

O hardware selecionado para a estação de medidas é o SoC nRF51822, produzido
pela Nordic Semiconductor. Este SoC conta com um processador ARM Cortex M0 de 32
bits e com um transceiver de 2.4GHz multiprotocolo, sendo capaz de operar com
os protocolos Bluetooth e ANT.\cite{nRF51ProdSpec}

\subsubsubsection{Estratégia para otimização de consumo energético}
Com a finalidade de otimizar o consumo energético foi adotada a estratégia de
manter o microcontrolador em modo deep sleep sempre que possível, sendo o mesmo
acordado somente através de interrupções para a realização das tarefas de rádio
e de leitura de sensores. Durante os períodos em sleep todos os sensores devem
estar num estado de baixo consumo de energia.

Ainda na otimização do consumo, a estratégia adotada para a transmissão dos
dados dos sensores foi concepção do dispositivo operando como um BLE advertiser
pois desta forma se reduz o período no qual há a necessidade do rádio, e
somente durante a configuração o dispositivo opera como um BLE Slave,
permitindo que uma conexão seja estabelecida e que os serviços e
características presentes sejam acessíveis.

\subsubsubsection{Bluetooth Advertiser}
No modo advertiser, o dispositivo transmite a informação de que este aceita o
estabelecimento de conexões com centrais BLE. Além disso, estão presentes nos
pacotes transmitidos as leituras realizadas pelos sensores.

As informações dos sensores são codificadas como payload do pacote seguindo a
seguinte estrutura de dados [ESTRUTURA]. Como formato de payload, optou-se por
utilizar o Manufacturer Specific Data, evitando assim possíveis conflitos com
pacotes BLE definidos pela especificação.

\lstinputlisting{../Development/ble-sensor-station/src/libs/ble/advertising/ble_adv_frame.h}

Os pacotes MSD são definidos na especificação suplementar, sendo o data type
definido como  

-> Utilização do msd
A transmissão dos pacotes de advertising são feitos através do formato de
Manufacturer Specific Data, desta garante-se a compatibilidade dos pacotes
transmitidos por este dispositivo com a especificação sem causar erro

Codificação do pacote de advertising

\subsubsubsection{Bluetooth Slave}
No modo slave, o dispostivo possui um serviço Bluetooth para cada sensor, com características que permitem a configuração dos sensores. 

De
Definição dos serviços



%End of Sensor Station
\subsubsection{Gateway Bluetooth - Internet}

\subsubsubsection{Especificação}

\subsubsubsection{Bluetooth Observer}

\subsubsubsection{Conectividade}

%end of BLE-Internet gateway
\subsubsection{Infraestrutura online}

\subsubsubsection{Servidor para recepção}

\subsubsubsection{Armazenamento e Visualização}
%end of infrastructure
%end of project
\subsection{Implementação}

\subsubsection{Estação de Medidas Bluetooth Low Energy}

A estação de medidas foi desenvolvida utilizando o SoC NRF51822, produzido pela Nordic
-> Hardware
-> Microcontrolador
-> Sensores		

\subsubsubsection{Ambiente de desenvolvimento}
O ambiente de desenvolvimento foi baseado no Software Development Kit oferecido
pelo fabricante do microcontrolador, o nRF5 SDK 12.3. Este SDK acelera o
desenvolvimento do projeto por diminuir a necessidade de acesso direto aos
registradores do microcontrolador para a operação de hardware, permitindo uma
programação em nível mais elevado.
 
O projeto de firmware foi feito utilizando-se Makefile e o toolchain ARM-GCC,
permitindo assim o desenvolvimento e compilação do projeto de forma
independente de IDE.

A IDE utilizada ao para o desenvolvimento foi o Eclipse IDE versão Oxygen. Além
disso, foram necessárias as ferramentas NRFJPROG, software desenvolvido pelo
fabricante do microcontrolador para realizar tarefas como gravação, leitura,
apagar memória, etc., e a ferramenta Segger J-Link, como interface entre o
computador e o microcontrolador para as operações do NRFJPROG e para a
depuração do firmware.
 
\subsubsubsection{Firmware}

Estratégia de firmware orientado a eventos
Linhas gerais sobre firmware para sensores
Serviços ble
ADV Encoder/decoder

%End of Sensor Station

\subsubsection{Gateway Bluetooth - Internet}
			-> Hardware
			-> Ambiente de desenvolvimento
%end of BLE-Internet gateway

\subsubsection{Infraestrutura online}
-> Ambiente de desenvolvimento	
-> Coleta e armazenamento de dados
%end of infrastructure
%end of implementation

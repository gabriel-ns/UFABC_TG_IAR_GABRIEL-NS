\section{Introdução}

% O que é BLE

% O que o BLE faz para ser Low Power

% Parâmetros do BLE

% Funcionamento fisico do BLE

% Arquitetura do BLE: GAP, GATT, ETC

% Camada mais baixa da aplicação: o GAP

% Quatro roles do BLE

% Aplicações do BLE

O Bluetooth Low Energy, ou BLE, é uma das formas de operação previstas na
especificação da versão 4.0 do Bluetooth. Junto com o BLE, o documento também
especifica a operação do tipo Basic Rate, compatibilizando a versão 4.0 com as
versões anteriores.


% Descrição do meio físico
O BLE opera na frequência de 2.4GHz utilizando 40 canais de 2MHz com
frequências de centro de 2402MHz a 2480MHz.Existem dois tipos de transmissão: a
transmissão de dados, que é feita em 37 dos canais disponíveis com a capacidade
de transmissão de 1Mbit/s empregando a técnica de Frequency-Hopping Spread
Spectrum para combater interferências, alterando a todo instante a frequência
da portadora numa sequência pseudoaleatória conhecida apenas entre o
transmissor e receptor; e o Advertising, que é feito nos outros 3 canais
restantes com as frequências de 2402MHz, 2426MHz e 2480MHz.

Advertising é a ação de transmitir pacotes de dados de forma pública para todos
os dispositivos capazes de recebê-los, com a finalidade principal de indicar a
presença do dispositivo no local.
Os canais de Advertising possuem, frequências estratégicas para evitar
interferências causadas pela coexistência do BLE com redes Wi-Fi.

% Camadas e arquitetura do BLE

% Descrição dos papéis

No BLE os dispositivos podem assumir 4 funções diferentes. São elas:
Broadcaster, Observer, Peripheral e Central.

%% definir GAP
Na função de Broadcaster, o dispositivo opera apenas transmitindo pacotes de
dados periodicamente utilizando o GAP. Estes pacotes de dados, além de indicar
a presença do dispositivo, contém os dados que são formatados de acordo com as
especificações do BLE.

Já na função de Observer o dispositivo opera apenas lendo dados Nestas duas
funções o fluxo de dados é unidirecional.

Na função de Peripheral o dispositivo é capaz de estabelecer conexões com
outros dispositivos, porém operando num modo slave sendo controlado através de
um dispositivo master. A quarta função é a central Bluetooth, que pode operar tanto com.


% O Bluetooth Low Energy, ou BLE, é uma das formas de operação previstas na
% especificação da versão 4.0 do Bluetooth.
% Junto com o BLE, o documento também especifica a operação do tipo Basic Rate,
% compatibilizando a versão 4.0 com as versões anteriores.


% Topologia:

% No BLE os dispositivos podem assumir 4 funções diferentes. São elas: Broadcaster, Observer, Peripheral e Central.
% Na função de Broadcaster, o dispositivo opera apenas transmitindo dados. Já na função de Observer o dispositivo opera apenas lendo dados Nestas duas funções o fluxo de dados é unidirecional.
% Na função de Peripheral o dispositivo é capaz de estabelecer conexões com outros dispositivos, porém operando num modo slave sendo controlado através de um dispositivo master. A quarta função é a central Bluetooth, que pode operar tanto com.

% Operação:

% O rádio do BLE opera na frequência de 2.4GHz utilizando a técnica de frequency hopping para combater interferência de muitas portatoras FHSS.

% São utilizados dois tipos de acesso: FDMA e TDMA. Existem 40 canais físicos separados por 2MHz utilizando a técnica FDMA. Entre esses canais, 37 são usados como canais de dados e 3 são canais utilizados para advertising.

% Aplicações:



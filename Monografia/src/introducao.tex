\section{Introdução}

O Bluetooth Low Energy, ou BLE, é uma das formas de operação previstas na especificação da versão 4.0 do Bluetooth. Junto com o BLE, o documento também especifica a operação do tipo Basic Rate, compatibilizando a versão 4.0 com as versões anteriores.	


Topologia:

No BLE os dispositivos podem assumir 4 funções diferentes. São elas: Broadcaster, Observer, Peripheral e Central.
Na função de Broadcaster, o dispositivo opera apenas transmitindo dados. Já na função de Observer o dispositivo opera apenas lendo dados Nestas duas funções o fluxo de dados é unidirecional.
Na função de Peripheral o dispositivo é capaz de estabelecer conexões com outros dispositivos, porém operando num modo slave sendo controlado através de um dispositivo master. A quarta função é a central Bluetooth, que pode operar tanto com.

Operação:

O rádio do BLE opera na frequência de 2.4GHz utilizando a técnica de frequency hopping para combater interferência de muitas portatoras FHSS.

São utilizados dois tipos de acesso: FDMA e TDMA. Existem 40 canais físicos separados por 2MHz utilizando a técnica FDMA. Entre esses canais, 37 são usados como canais de dados e 3 são canais utilizados para advertising.

Aplicações:



\section{Introdução}

% Bluetooth Overview

% O que é BLE

% O que o BLE faz para ser Low Power

% Parâmetros do BLE

% Descrição do meio físico
O BLE opera na frequência de 2.4GHz utilizando 40 canais de 2MHz com
frequências de centro de 2402MHz a 2480MHz. Existem dois tipos de transmissão: a
transmissão de dados, que é feita em 37 dos canais disponíveis com a capacidade
de transmissão de 1Mbit/s; e o Advertising, que é feito nos outros 3 canais
restantes com as frequências de 2402MHz, 2426MHz e 2480MHz.\cite{ble4core}

A tabela \ref{tab:adv_channel_list} mapeia os canais de rádiofrequência para os canais
utilizados pelo BLE com suas respectivas identificações e finalidades.

\begin{center}
	\centering 
	\includegraphics[width=0.8\linewidth]{adv_ch_map.png}
	\captionof{table}{Mapeamento dos canais de RF para os canais do BLE}
	\label{tab:adv_channel_list}
\end{center} 

Advertising é a ação de transmitir pacotes de dados de forma pública para todos
os dispositivos capazes de recebê-los, com a finalidade principal de indicar a
presença do dispositivo no local.

Os canais de Advertising possuem, frequências estratégicas para evitar
interferências causadas pela coexistência do BLE com redes Wi-Fi, como 
mostra a figura \ref{fig:wifi_coexist}

\begin{center}
	\centering 
	\includegraphics[width=0.6\linewidth]{ble-advertising-channels-spectrum.png}
	\captionof{table}{Coexistência entre BLE e Wi-Fi}
	\label{tab:adv_channel_list}
\end{center} 

% GAP
    %Profiles

Na função de Broadcaster, o dispositivo opera apenas transmitindo pacotes de dados periodicamente nos três canais de advertising. 
Estes pacotes de dados, além de indicar a presença do dispositivo, contém os dados que são formatados de acordo com as especificações do BLE. 
Existem dois tipos de pacotes que o Broadcaster envia, que pode ser o advertising, pacote transmitido periodicamente, ou scan response,
que é o pacote transmitido quando um Observer detecta um advertising e solicita outro pacote com mais informações.

Já na função de Observer o dispositivo opera apenas lendo dados Nestas duas
funções o fluxo de dados é unidirecional.
    
Na função de Peripheral o dispositivo é capaz de estabelecer conexões com
outros dispositivos, porém operando num modo slave sendo controlado através de
um dispositivo master. A quarta função é a central Bluetooth, que pode operar tanto com.

    % Serviços e caracteríscas

    %address

% Link layer

% Aplicações do BLE

O Bluetooth Low Energy, ou BLE, é uma das formas de operação previstas na
especificação da versão 4.0 do Bluetooth. Junto com o BLE, o documento também
especifica a operação do tipo Basic Rate, compatibilizando a versão 4.0 com as
versões anteriores.

%% definir GAP

% O Bluetooth Low Energy, ou BLE, é uma das formas de operação previstas na
% especificação da versão 4.0 do Bluetooth.
% Junto com o BLE, o documento também especifica a operação do tipo Basic Rate,
% compatibilizando a versão 4.0 com as versões anteriores.


% Topologia:

% No BLE os dispositivos podem assumir 4 funções diferentes. São elas: Broadcaster, Observer, Peripheral e Central.
% Na função de Broadcaster, o dispositivo opera apenas transmitindo dados. Já na função de Observer o dispositivo opera apenas lendo dados Nestas duas funções o fluxo de dados é unidirecional.
% Na função de Peripheral o dispositivo é capaz de estabelecer conexões com outros dispositivos, porém operando num modo slave sendo controlado através de um dispositivo master. A quarta função é a central Bluetooth, que pode operar tanto com.

% Operação:

% O rádio do BLE opera na frequência de 2.4GHz utilizando a técnica de frequency hopping para combater interferência de muitas portatoras FHSS.

% São utilizados dois tipos de acesso: FDMA e TDMA. Existem 40 canais físicos separados por 2MHz utilizando a técnica FDMA. Entre esses canais, 37 são usados como canais de dados e 3 são canais utilizados para advertising.

% Aplicações:



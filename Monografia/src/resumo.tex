\section{Resumo}
% Revisão sobre o funcionamento do BLE

Este trabalho estuda as características principais do Bluetooth Low Energy da
perspectiva do desenvolvimento de aplicações, propondo uma metodologia para a
elaboração de projetos com Bluetooth Low Energy e realizando um projeto de
exemplo seguindo a metodologia proposta. 

O projeto desenvolvido é de uma estação de monitoramento de variáveis ambientais
com medidas de temperatura, umidade relativa do ar, pressão atmosférica e
luminosidade que transmite as medidas de forma sem fio através do Bluetooth Low
Energy, sendo as medidas transmitidas captadas por outro dispositivo que envia
as medidas para a internet. Esta implementação foi feita utilizando o 
microcontrolador nRF51822 da Nordic Semiconductor para a leitura das variáveis,
o microcontrolador Espressif ESP32 para receber as medidas via bluetooth e
enviar para a internet, e o sistema da Amazon AWS IoT Core para receber estes
dados na internet através  do protocolo MQTT. 

É montado e mostrado também o setup de um ambiente de desenvolvimento gratuito
e sem limitações de tamanho de programa ou de códigos fonte para os dois
microcontroladores.

%conclusão
\documentclass[12pt,a4paper,draft]{article}
\usepackage[brazil]{babel}
\usepackage[utf8]{inputenc}
\usepackage[final]{listings}
% \usepackage{listings}
\usepackage{amsmath}
\usepackage{amsfonts}
\usepackage{amssymb}
\usepackage{array}
\usepackage[final]{graphicx}
\usepackage[left=3.00cm, right=2.00cm, top=3.00cm, bottom=2.00cm]{geometry}
\usepackage{cite}
\usepackage{indentfirst} 
\usepackage{tabularx,ragged2e,booktabs,caption}
\usepackage{multirow}
\usepackage{tikz}
\usepackage{hyperref}
\usepackage{color}
\usepackage[all]{tcolorbox}
\usepackage{setspace}
\usepackage{enumitem}
\usepackage[acronym]{glossaries}
\usepackage{longtable}
\usepackage{nomencl}
\makenomenclature

%% This removes the main title:
\renewcommand{\nomname}{}
%% this modifies item separation:
\setlength{\nomitemsep}{8pt}
%% this part defines the groups:
%----------------------------------------------
\usepackage{etoolbox}
\renewcommand\nomgroup[1]{%
  \item[\Large\bfseries
  \ifstrequal{#1}{N}{Nomenclaturas}{%
  \ifstrequal{#1}{A}{Lista de Abreviações}{}}%
]\vspace{10pt}} % this is to add vertical space between the groups.
%----------------------------------------------


\newenvironment{packed_itemize}{
\begin{itemize}
  \setlength{\itemsep}{0.5pt}
  \setlength{\parskip}{0pt}
  \setlength{\parsep}{0pt}
}{\end{itemize}}

\graphicspath{ {img/} } 

\renewcommand{\baselinestretch}{1.5} 

\addto\captionsbrazil{%
  \renewcommand*{\lstlistlistingname}{Lista de Códigos Fonte}% 
  \renewcommand*{\lstlistingname}{Código Fonte} %
}
\definecolor{mygreen}{rgb}{0.0,0.4,0.0}
\definecolor{mygray}{rgb}{0.5,0.5,0.5}
\definecolor{mymauve}{rgb}{0.58,0,0.82} 
\definecolor{backcolour}{rgb}{0.95,0.95,0.92}

\lstset{ %
	backgroundcolor=\color{backcolour},   % choose the background color; you must add
	stringstyle=\color{mymauve},     % string literal style
	keywordstyle=\color{blue},       % keyword style
	numberstyle=\tiny\color{mygray}, % the style that is used for the line-numbers
	commentstyle=\color{mygreen},    % comment style
	% \usepackage{color} or \usepackage{xcolor}; should come as last argument
	basicstyle=\footnotesize\ttfamily,        % the size of the fonts that are used for the code
	breakatwhitespace=false,         % sets if automatic breaks should only happen at whitespace
	breaklines=false,                 % sets automatic line breaking
	captionpos=b,                    % sets the caption-position to bottom
	deletekeywords={...},            % if you want to delete keywords from the given language
	escapeinside={\%*}{*)},          % if you want to add LaTeX within your code
	extendedchars=true,              % lets you use non-ASCII characters; for 8-bits encodings only, does not work with UTF-8
% 	frame=single,	                   % adds a frame around the code
	keepspaces=true,                 % keeps spaces in text, useful for keeping indentation of code (possibly needs columns=flexible)
	language=C,                      % the language of the code
	numbers=left,                    % where to put the line-numbers; possible values are (none, left, right)
	numbersep=5pt,                   % how far the line-numbers are from the code
	rulecolor=\color{black},         % if not set, the frame-color may be changed on line-breaks within not-black text (e.g. comments (green here))
	showspaces=false,                % show spaces everywhere adding particular underscores; it overrides 'showstringspaces'
	showstringspaces=false,          % underline spaces within strings only
	showtabs=false,                  % show tabs within strings adding particular underscores
	stepnumber=1,                    % the step between two line-numbers. If it's 1, each line will be numbered
	tabsize=4,	                     % sets default tabsize to 4 spaces
	title=\lstname,                   % show the filename of files included with \lstinputlisting; also try caption instead of title
	morekeywords={*, uint32_t, int32_t, uint16_t, int16_t, uint8_t, int8_t,
	packed, sensor_err_code_t,	sensor_ctrl_data_t,	sensor_ctrl_cfg_t, tsl_gain_t,
	tsl_int_time_t, htu_resolution_t, bmp_pwr_mode_t, sensor_type_t,
	sensor_state_t,esp_ble_gap_cb_param_t, bool,
	esp_gap_ble_cb_event_t,ble_adv_data_t, esp_bt_controller_config_t, memcmp...},
	}

\newcommand\dblquote[1]{\textquotedblleft #1\textquotedblright}
\makeatletter
\newcommand\subsubsubsection{\@startsection{paragraph}{4}{\z@}%
            {-2.5ex\@plus -1ex \@minus -.25ex}%
            {1.25ex \@plus .25ex}%
            {\normalfont\normalsize\bfseries}}
\makeatother
\setcounter{secnumdepth}{4} % how many sectioning levels to assign numbers to
\setcounter{tocdepth}{4}    % how many sectioning levels to show in ToC

\newcommand{\institution}{UNIVERSIDADE FEDERAL DO ABC}
\newcommand{\course}{ENGENHARIA DE INSTRUMENTAÇÃO, AUTOMAÇÃO E ROBÓTICA}
\newcommand{\local}{SANTO ANDRÉ}
\newcommand{\theauthor}{GABRIEL NASCIMENTO DOS SANTOS}
\newcommand{\thetitle}{DESENVOLVIMENTO DE REDE DE SENSORES BLUETOOTH LOW ENERGY
CONECTADOS À NUVEM}
\newcommand{\thedate}{2018}

\begin{document}
		
	\begin{titlepage}
	\vfill
	\begin{center}
		{\large \textbf{\institution}} \\ \course \\[3cm]
		
		{\large \textbf{\theauthor}}\\[4cm]
		
		
		{\Large \thetitle}\\[4cm]
		

		\vspace{5cm}
		
		\large \textbf{\local}
		
		\large \textbf{\thedate}
	\end{center}
\end{titlepage}

	  \begin{center}
  \thispagestyle{empty}
   {\large \textbf{\theauthor}} \\[6cm]



   {\Large \thetitle}\\[6cm]

   \hspace{.25\textwidth} %posiciona a minipage
   \begin{minipage}{.7\textwidth}
   \large Relatório apresentado à disciplina “Trabalho de
   Graduação I” do curso de Engenharia
   de Instrumentação, Automação e Robótica da Universidade Federal do
   ABC
   \\[1cm]
   Orientador: Prof. Dr. Carlos Alberto dos Reis Filho
  \end{minipage}
  \vfill
  

\vspace{2cm}

\large \textbf{\local}

\large \textbf{\thedate}
\end{center}
\newpage

	
	\section{Resumo}
% Revisão sobre o funcionamento do BLE

Este trabalho estuda as características principais do Bluetooth Low Energy da
perspectiva do desenvolvimento de aplicações, propondo uma metodologia para a
elaboração de projetos com Bluetooth Low Energy e realizando um projeto de
exemplo seguindo a metodologia proposta. 

O projeto desenvolvido é de uma estação de monitoramento de variáveis ambientais
com medidas de temperatura, umidade relativa do ar, pressão atmosférica e
luminosidade que transmite as medidas de forma sem fio através do Bluetooth Low
Energy, sendo as medidas transmitidas captadas por outro dispositivo que envia
as medidas para a internet. Esta implementação foi feita utilizando o 
microcontrolador nRF51822 da Nordic Semiconductor para a leitura das variáveis,
o microcontrolador Espressif ESP32 para receber as medidas via bluetooth e
enviar para a internet, e o sistema da Amazon AWS IoT Core para receber estes
dados na internet através  do protocolo MQTT. 

É montado e mostrado também o setup de um ambiente de desenvolvimento gratuito
e sem limitações de tamanho de programa ou de códigos fonte para os dois
microcontroladores.

%conclusão
	 
	\newpage
	\listoffigures
	 
	\newpage
	\listoftables
	
	\newpage
	\lstlistoflistings
	
	\newpage
	\nomenclature[A]{\textbf{IMO}}{in my opinion}
\nomenclature[A]{\textbf{OP}}{original poster}
	\printnomenclature[2cm]  

	\newpage
	\tableofcontents

	\section{Introdução}

% Bluetooth Overview

% O que é BLE

% O que o BLE faz para ser Low Power

% Parâmetros do BLE

% Descrição do meio físico
O BLE opera na frequência de 2.4GHz utilizando 40 canais de 2MHz com
frequências de centro de 2402MHz a 2480MHz. Existem dois tipos de transmissão: a
transmissão de dados, que é feita em 37 dos canais disponíveis com a capacidade
de transmissão de 1Mbit/s; e o Advertising, que é feito nos outros 3 canais
restantes com as frequências de 2402MHz, 2426MHz e 2480MHz.\cite{ble4core}

A tabela \ref{tab:adv_channel_list} mapeia os canais de rádiofrequência para os canais
utilizados pelo BLE com suas respectivas identificações e finalidades.

\begin{center}
	\centering 
	\includegraphics[width=0.8\linewidth]{adv_ch_map.png}
	\captionof{table}{Mapeamento dos canais de RF para os canais do BLE}
	\label{tab:adv_channel_list}
\end{center} 

Advertising é a ação de transmitir pacotes de dados de forma pública para todos
os dispositivos capazes de recebê-los, com a finalidade principal de indicar a
presença do dispositivo no local.

Os canais de Advertising possuem, frequências estratégicas para evitar
interferências causadas pela coexistência do BLE com redes Wi-Fi, como 
mostra a figura \ref{fig:wifi_coexist}

\begin{center}
	\centering 
	\includegraphics[width=0.6\linewidth]{ble-advertising-channels-spectrum.png}
	\captionof{table}{Coexistência entre BLE e Wi-Fi}
	\label{tab:adv_channel_list}
\end{center} 

% GAP
    %Profiles

Na função de Broadcaster, o dispositivo opera apenas transmitindo pacotes de dados periodicamente nos três canais de advertising. 
Estes pacotes de dados, além de indicar a presença do dispositivo, contém os dados que são formatados de acordo com as especificações do BLE. 
Existem dois tipos de pacotes que o Broadcaster envia, que pode ser o advertising, pacote transmitido periodicamente, ou scan response,
que é o pacote transmitido quando um Observer detecta um advertising e solicita outro pacote com mais informações.

Já na função de Observer o dispositivo opera apenas lendo dados Nestas duas
funções o fluxo de dados é unidirecional.
    
Na função de Peripheral o dispositivo é capaz de estabelecer conexões com
outros dispositivos, porém operando num modo slave sendo controlado através de
um dispositivo master. A quarta função é a central Bluetooth, que pode operar tanto com.

    % Serviços e caracteríscas

    %address

% Link layer

% Aplicações do BLE

O Bluetooth Low Energy, ou BLE, é uma das formas de operação previstas na
especificação da versão 4.0 do Bluetooth. Junto com o BLE, o documento também
especifica a operação do tipo Basic Rate, compatibilizando a versão 4.0 com as
versões anteriores.

%% definir GAP

% O Bluetooth Low Energy, ou BLE, é uma das formas de operação previstas na
% especificação da versão 4.0 do Bluetooth.
% Junto com o BLE, o documento também especifica a operação do tipo Basic Rate,
% compatibilizando a versão 4.0 com as versões anteriores.


% Topologia:

% No BLE os dispositivos podem assumir 4 funções diferentes. São elas: Broadcaster, Observer, Peripheral e Central.
% Na função de Broadcaster, o dispositivo opera apenas transmitindo dados. Já na função de Observer o dispositivo opera apenas lendo dados Nestas duas funções o fluxo de dados é unidirecional.
% Na função de Peripheral o dispositivo é capaz de estabelecer conexões com outros dispositivos, porém operando num modo slave sendo controlado através de um dispositivo master. A quarta função é a central Bluetooth, que pode operar tanto com.

% Operação:

% O rádio do BLE opera na frequência de 2.4GHz utilizando a técnica de frequency hopping para combater interferência de muitas portatoras FHSS.

% São utilizados dois tipos de acesso: FDMA e TDMA. Existem 40 canais físicos separados por 2MHz utilizando a técnica FDMA. Entre esses canais, 37 são usados como canais de dados e 3 são canais utilizados para advertising.

% Aplicações:



	
	\section{Objetivos}

%TODO Abordagem didática do Bluetooth
O objetivo deste trabalho é realizar uma abordagem didática sobre o protocolo Bluetooth Low Energy

%TODO desenvolvimento de projeto de referência para o desenvolvimento de
% aplicações que envolvam o bluetooth

%
Este projeto tem por objetivo realizar um estudo sobre a tecnologia Bluetooth
Low Energy com foco em aplicações de internet das coisas para monitoramento
wireless, desenvolvendo uma estação de medidas capaz de ler variáveis do
ambiente e transmitir dados sem fio utilizando baterias como fonte de energia,
e uma estação com acesso à internet para recepção e disponibilização destes
dados na rede.s

	
	\section{Metodologia}

A metodologia utilizada neste trabalho é uma proposta de metodologia de projeto
para aplicações utilizando Bluetooth Low Energy. 

Para exemplificar a utilização do método, o projeto feito é implementado numa
plataforma de prototipagem para a validação da metodologia.

\subsection{Projeto de aplicação com Bluetooth Low Energy}

A primeira etapa do projeto do dispositivo BLE consiste na definição de todas
tarefas que deverão ser realizadas pelo dispositivo. Definidas as tarefas, é
escolhida uma técnica para a troca de informações com o dispositivo, isso
implica em definir quais tarefas o dispositivo irá realizar como broadcaster,
como slave, como scanner, como central e quais tarefas são independentes do
bluetooth, como por exemplo acesso a senores ou atuadores.

O projeto das tarefas relacionadas ao bluetooth pode ser feito sem especificar
os componentes de hardware necessário, porém é interessante selecionar os
componentes nesta etapa do projeto. Desta forma é possível saber de antemão
quais são os recursos e informações que podem ser disponibilizados.

Ao trabalhar com projetos que envolvem baixo consumo de energia ou projetos com
energia limitada, o consumo energético dos componentes se torna uma
característica importante, necessitando a análise dos datasheets de cada
componente buscando sempre um consumo energético aceitável.

Em relação ao bluetooth deve-se projetar cada forma de operação do dispositivo e
os parâmetros comuns de operação.

No projeto da operação como broadcaster deve-se definir como serão codificadas
as informações atribuídas ao broadcaster nos pacotes de advertising. Além
disso, a especificação suplementar especifica uma série de informações
pré-definidas que podem ser transmitidas no mesmo pacote de advertising emitido
pelo broadcaster\cite{ble4sup}. Pode ser necessária também a especificação do
pacote de scan response, que deve seguir as mesmas especificações do
advertising. Ao término desta definição, é possível prever quantos tipos
diferentes de pacotes o dispositivo deverá transmitir.

No projeto da operação como slave, deve-se definir todos os serviços bluetooth
que o dispositivo deve oferecer, bem como as suas respectivas características. 
No serviço, define-se o UUID único do serviço. Já para as características, é
necessário definir o UUID, os tamanhos mínimos e máximos da característica em
bytes, as permissões de leitura e escrita, o envio de notificações, o envio de
alarmes e os descritores das características.

No projeto da operação como observer, deve-se definir a janela de
amostragem, o intervalo entre amostragens e quais são os pacotes que devem ser
processados pelo dispositivo.

No projeto da operação como central, deve-se definir quais são os dispositivos
alvo da conexão, quais são os serviços e caracteríticas que o dispositivo deve
acessar, ler e escrever, bem como definir quais são os dados escritos. É
importante notar que para operar como central o dispositivo também deve operar
como observer com a finalidade de identificar os dispositivos alvo. É
necessário também saber quais são os serviços disponíveis nos dispositivos alvo.

Pode ser necessária uma ou mais iterações entre as etapas de projeto com a
finalidade compatibilizar todos os elementos do projeto.

\subsection{Implementação e verificação do projeto}

A implementação do dispositivo se deu através da construção do hardware
necessário para a integração dos componentes e da elaboração do software
embarcado responsável pelo gerenciamento dos sensores e de todas as operações
realizadas pelo bluetooth, e pela verificação das funcionalidades do
dispositivo.

\subsubsection{Montagem do Hardware}
\subparagraph{Material Utilizado}

\begin{itemize}[noitemsep]
  \item Kit de desenvolvimento BLE400 para nRF51822
  \item Kit de desenvolvimento ESP32-DevKitC para ESP32
  \item Breakout Board do sensor BMP180
  \item Breakout Board do sensor HTU21D
  \item Breakout Board do sensor TSL2561
  \item Gravador OB JLink V8
  \item Placa padrão
  \item Estanho para solda liga 60\%Sn/40\%Pb
\end{itemize}

A construção do hardware ocorreu nas seguintes etapas: compra dos
componentes, definição dos pinos da comunicação entre os componentes e soldagem
dos componentes em placa padrão com conectores para efetuar as ligações.


\subsection{Elaboração do Software Embarcado}

\subparagraph{Material Utilizado}
\begin{itemize}[noitemsep]
  \item Software Eclipse IDE versão Oxygen
  \item NRF5 Software Development Kit versão 12.3
  \item Espressif IoT Development Framework versão 3.1
  \item Serviço em núvem Amazon AWS IoT Core
  \item Make
  \item GNU Arm Embedded Toolchain versão 6.3.1
  \item Ferramenta NRFJPROG versão 9.7.2
  \item SEGGER JLink Software versão 6.12f
\end{itemize}

A elaboração do software embarcado se deu nas seguintes etapas: teste do
ambiente de desenvolvimento com os exemplos do fabricante da plataforma,
definição dos componentes de software e arquitetura, elaboração e testes das
bibliotecas de acesso aos sensores em linguagem C com base nos datasheets dos
componentes, elaboração dos demais componentes de software em linguagem C e
integração de todos os programas desenvolvidos.

\subsection{Verificação das Funcionalidades}

\subparagraph{Material Utilizado}
\begin{itemize}[noitemsep]
  \item Aplicativo para smartphone Android nRF Connect versão 4.19.1
  \item Software para terminal serial Minicom versão 2.7 para sistema
  operacional Linux
  \item Navegador de internet Mozilla Firefox versão 58.0.2
  \item Osciloscópio TEKTRONIX MDO-3104
  \item Medidor de corrente EE-QUIPMENT EE-203
\end{itemize}

Através do aplicativo nRF Connect foi possível testar as funcionalidades
relacionadas ao bluetooth broadcaster, observando e analisando os pacotes de
advertisement enviados, e ao bluetooth slave, estabelecendo a conexão,
realizando a leitura dos sensores, habilitando o streaming de dados e testando
novas configurações no dispositivo.

As funcionalidades do bluetooth observer foram testadas através do terminal
serial, enviando os pacotes recebidos através da porta UART para um computador
que mostra as mensagens recebidas.

O envio de mensagens para a internet foi testado através do dashboard online
oferecido pela Amazon AWS que permite o monitoramento do projeto através do
navegador de internet.

Também foram realizados testes de performance energética no dispositivo através
de um medidor de corrente do tipo shunt alimentando a placa, com a leitura do
medidor ligada a um osciloscópio digital.


	
	\section{Desenvolvimento}

\subsection{Projeto}

O projeto desenvolvido tem como funcionalidade principal a coleta de leituras
periódicas de sensores com a transmissão sem fio das variáveis medidas
utilizando o bluetooth low energy. Estes dados, quando recebidos, são enviados
para a internet.

Para tal, dividiu-se as tarefas em três sistemas diferentes: a estação de
medidas, responsável pela coleta e transmissão das medididas através do
bluetooth, o gateway BLE-Internet, responsável por captar os dados no ar e
enviá-los para a internet, e a infraestrutura online, que recebe a informação e
a armazena on-line.

\subsubsection{Estação de Medidas Bluetooth Low Energy}

A estação de medidas foi projetada para atender os requisitos propostos,
visando o mínimo consumo energético para possibilitar a alimentação através de
baterias sem a necessidade de manutenções frequentes.
 
Como diferencial a estação permite a configuração de sua operação através de
serviços GATT do BLE, sendo possível alterar tanto as configurações de rádio
(potêcia e intervalo de transmissão) quanto os parâmetros de configuração dos
sensores (range, escala, intervalo entre medidas).

O hardware selecionado para a estação de medidas é o SoC nRF51822, produzido
pela Nordic Semiconductor. Este SoC conta com um processador ARM Cortex M0 de 32
bits e com um transceiver de 2.4GHz multiprotocolo, sendo capaz de operar com
os protocolos Bluetooth e ANT.\cite{nRF51ProdSpec}

\subsubsubsection{Estratégia para otimização de consumo energético}
Com a finalidade de otimizar o consumo energético foi adotada a estratégia de
manter o microcontrolador em modo deep sleep sempre que possível, sendo o mesmo
acordado somente através de interrupções para a realização das tarefas de rádio
e de leitura de sensores. Durante os períodos em sleep todos os sensores devem
estar num estado de baixo consumo de energia.

Ainda na otimização do consumo, a estratégia adotada para a transmissão dos
dados dos sensores foi concepção do dispositivo operando como um BLE advertiser
pois desta forma se reduz o período no qual há a necessidade do rádio, e
somente durante a configuração o dispositivo opera como um BLE Slave,
permitindo que uma conexão seja estabelecida e que os serviços e
características presentes sejam acessíveis.

\subsubsubsection{Bluetooth Advertiser}
No modo advertiser, o dispositivo transmite a informação de que este aceita o
estabelecimento de conexões com centrais BLE. Além disso, estão presentes nos
pacotes transmitidos as leituras realizadas pelos sensores.

As informações dos sensores são codificadas como payload do pacote seguindo a
seguinte estrutura de dados [ESTRUTURA]. Como formato de payload, optou-se por
utilizar o Manufacturer Specific Data, evitando assim possíveis conflitos com
pacotes BLE definidos pela especificação.

\lstinputlisting{../Development/ble-sensor-station/src/libs/ble/advertising/ble_adv_frame.h}

Os pacotes MSD são definidos na especificação suplementar, sendo o data type
definido como  

-> Utilização do msd
A transmissão dos pacotes de advertising são feitos através do formato de
Manufacturer Specific Data, desta garante-se a compatibilidade dos pacotes
transmitidos por este dispositivo com a especificação sem causar erro

Codificação do pacote de advertising

\subsubsubsection{Bluetooth Slave}
No modo slave, o dispostivo possui um serviço Bluetooth para cada sensor, com características que permitem a configuração dos sensores. 

De
Definição dos serviços



%End of Sensor Station
\subsubsection{Gateway Bluetooth - Internet}

\subsubsubsection{Especificação}

\subsubsubsection{Bluetooth Observer}

\subsubsubsection{Conectividade}

%end of BLE-Internet gateway
\subsubsection{Infraestrutura online}

\subsubsubsection{Servidor para recepção}

\subsubsubsection{Armazenamento e Visualização}
%end of infrastructure
%end of project
\subsection{Implementação}

\subsubsection{Estação de Medidas Bluetooth Low Energy}

A estação de medidas foi desenvolvida utilizando o SoC NRF51822, produzido pela Nordic
-> Hardware
-> Microcontrolador
-> Sensores		

\subsubsubsection{Ambiente de desenvolvimento}
O ambiente de desenvolvimento foi baseado no Software Development Kit oferecido
pelo fabricante do microcontrolador, o nRF5 SDK 12.3. Este SDK acelera o
desenvolvimento do projeto por diminuir a necessidade de acesso direto aos
registradores do microcontrolador para a operação de hardware, permitindo uma
programação em nível mais elevado.
 
O projeto de firmware foi feito utilizando-se Makefile e o toolchain ARM-GCC,
permitindo assim o desenvolvimento e compilação do projeto de forma
independente de IDE.

A IDE utilizada ao para o desenvolvimento foi o Eclipse IDE versão Oxygen. Além
disso, foram necessárias as ferramentas NRFJPROG, software desenvolvido pelo
fabricante do microcontrolador para realizar tarefas como gravação, leitura,
apagar memória, etc., e a ferramenta Segger J-Link, como interface entre o
computador e o microcontrolador para as operações do NRFJPROG e para a
depuração do firmware.
 
\subsubsubsection{Firmware}

Estratégia de firmware orientado a eventos
Linhas gerais sobre firmware para sensores
Serviços ble
ADV Encoder/decoder

%End of Sensor Station

\subsubsection{Gateway Bluetooth - Internet}
			-> Hardware
			-> Ambiente de desenvolvimento
%end of BLE-Internet gateway

\subsubsection{Infraestrutura online}
-> Ambiente de desenvolvimento	
-> Coleta e armazenamento de dados
%end of infrastructure
%end of implementation

	
	\section{Resultados}

\subsection{Estação de Medidas}

\subsubsection{Leitura dos sensores}

\subsubsection{Transmissão das leituras - Advertising}

\subsubsection{Serviços Bluetooth}

\subsection{Gateway Bluetooth - Internet}

\subsubsection{Bluetooth Observer}

\subsubsection{Decodificação dos pacotes}

\subsubsection{Conexão com o AWS IoT Core}

\subsubsection{Publicação na Internet}

\subsection{Nuvem}

\subsubsection{Comunicação com o Gateway Bluetooth - Wifi}

\subsubsection{Armazenamento}

\subsubsection{Acesso aos dados}

\subsection{Performance energética}

\subsubsection{Medidas de consumo de energia}

\subsubsection{Advertising}

\subsubsection{Conexão Bluetooth}

\subsubsection{Leitura dos sensores}

\subsubsection{Standby}
	
	\section{Conclusões e Perspectivas}

% a metodologia se mostrou eficaz


% estação de medidas foi completamente funcional
A estação de medidas desenvolvida se mostrou funcional para o propósito. Numa
próxima etapa é possível trabalhar as questões relacionadas a eficiência
energética do dispositivo, estimativas de duração de bateria, e até mesmo
desenvolver um hardware próprio independente de kits de desenvolvimento.

Durante os testes foi possível notar também que os dados transmitidos pela
estação de medidas podem ser perdidos pelos dispositivos que os recebem,
especialmente nos smartphones em que não há acesso ao controle da janela de
amostragem. Este tipo de perde evidencia que o sistema atende os requisitos
apenas para monitoramento, porém é necessário elevar a confiabilidade da entrega
das medidas para utilizar esta técnica em sistemas de controle.


% dados chegaram na internet
	% há a possibilidade de perda de pacotes
	% custo da infraestrutura online

%PERSPECTIVAS

% que o trabalho sirva de referência
% que o trabalho contribua com o passo inicial para o desenvolvimento de mais
% aplicações com BLE
% contribuir com os primeros passos na programação de microcontroladores ARM de
% 32 bits


	
	\bibliographystyle{ieeetr}
	
	\bibliography{references.bib}
	
\end{document}